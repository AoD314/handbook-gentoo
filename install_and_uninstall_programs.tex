\subsection{Установка и удаление программ}
\subsubsection{Обновление дерева портежей}
Перед установкой программ обновите локальный репозиторий пакетов. Обновлять репозиторий следует не чаще 1 раза в день.
Для обновления достаточно выполнить с правами пользователя root команду:
\begin{verbatim}
eix-sync
\end{verbatim}

Программа обновит дерево портежей, оверлей Calculate, а затем синхронизирует свой локальный кэш, используемый при поиске программ.

\subsubsection{Поиск программ}
В программу emerge включен инструмент поиска программ, однако вы можете воспользоваться более быстрым инструментом - программой eix.
Пример:
\begin{verbatim}
eix freeciv
eix -S game
\end{verbatim}
В первом случае поиск производится по названию пакета, во втором - по описанию.

\subsubsection{Установка и удаление}
Установка и удаление программ производится при помощи программы emerge. При установке новой программы сначала определяется необходимость в установке дополнительных пакетов (зависимостей), а затем скачиваются и компилируются исходные тексты. Помните, что для установки программ вам потребуется наличие интернета и некоторое время для компиляции. Программы, требующие значительного времени для компиляции (например, OpenOffice), распространяются в виде готовых к установке бинарных пакетов.
Пример установки игры Цивилизация и бинарного OpenOffice:
\begin{verbatim}
emerge -bk games-strategy/freeciv
emerge openoffice-bin
\end{verbatim}

Параметры bk создают локальный архив скомпилированного пакета, а при его наличии программа устанавливается из этого архива, минуя стадию компиляции. Инструкцию по работе с программой emerge можно прочесть здесь.
Пример удаления игры Цивилизация:
\begin{verbatim}
emerge -C games-strategy/freeciv
\end{verbatim}
Переменные DISTDIR и PKGDIR указывают путь к локальным папкам, в которых сохраняются исходные тексты программ и откомпилированные пакеты. Посмотреть значения этих (и многих других) переменных, используемых командой emerge, можно с помощью команды
emerge --info
