\documentclass[12pt, a4paper]{article}

\usepackage[utf8]{inputenc}
\usepackage[english, russian]{babel}
\usepackage[T2A]{fontenc}
\usepackage{xltxtra}
\usepackage{xecyr}

%\usepackage{polyglossia}  %% подключает пакет многоязыкой вёрстки
\setmainfont{Liberation Serif}  %% задаёт основной шрифт документа
\setsansfont{Liberation Sans}  %% задаёт шрифт без засечек
\setmonofont{Droid Sans Mono}  %% задаёт моноширинный шрифт
\defaultfontfeatures{Scale=MatchLowercase, Mapping=tex-text}  %% устанавливает поведение шрифтов по умолчанию
%\setdefaultlanguage[spelling=modern]{russian}  %% устанавливает язык по умолчанию
%\setotherlanguage{english}

\usepackage{cmap} % чтобы был поиск по pdf

% меняем размеры страницы
\usepackage{geometry}
\geometry{left=2.0cm}
\geometry{right=2.0cm}
\geometry{top=2.0cm}
\geometry{bottom=2.5cm}

% переопределяем окружение verbatim
\makeatletter 
\renewcommand*\verbatim@font{% 
\normalfont\ttfamily\small 
\hyphenchar\font\m@ne 
\let\do\do@noligs \verbatim@nolig@list 
} 
\makeatother

\renewcommand{\thesection}{\arabic{section}.} 
\renewcommand{\thesubsection}{\arabic{section}.\arabic{subsection}.} 
\renewcommand{\thesubsubsection}{\arabic{section}.\arabic{subsection}.\arabic{subsubsection}.} 

\begin{document}

\begin{titlepage}
\begin{center}
\vspace*{10cm}
{ \rm \Huge \textbf
	{
		\fontspec{Ubuntu}{print(gentoo.calculate\_linux)}
	}
}
\vfill 
{ \rm \large  \fontspec{Ubuntu}{2011}  }
\end{center}
\end{titlepage}

\tableofcontents

\newpage

\section{Установка. Обновление. Профили}

\subsection{Системные требования}
\subsection{Gentoo}

\subsubsection{Gentoo. Установка}
\subsubsection{Gentoo. Обновление}

\subsection{Calculate Linux}

\subsubsection{Calculate Linux. Установка}
\subsubsection{Calculate Linux. Обновление}

\subsection{Управление профилями пользователей}

\newpage 

\section{Работа с коммандной строкой}

\subsection{Поиск файлов}

\newpage

\section{Менеджер пакетов}

\subsection{Установка и удаление программ}

\subsection{Разрешение зависимостей}

\newpage
\section{Настройка}

\subsection{Настройка сети}

\subsection{Настройка автозапуска}

%\subsection{}
%\subsection{}
%\subsection{}
%\subsection{}
%\subsection{}
%\subsection{}
%\subsection{}

\subsection{Персональные настройки}

\subsubsection{Настройка bash}

\subsubsection{Настройка vim}

\newpage
\section{Мониторинг и производительность}

\subsection{Мониторинг}

\subsection{Оптиммизация eix}

\subsection{SSD}

\subsection{Распределенная компиляция}

\newpage

\section{Приложение}

\end{document}

