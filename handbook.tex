\documentclass[12pt, a4paper]{article}

\usepackage[utf8]{inputenc}
\usepackage[english, russian]{babel}
\usepackage[T2A]{fontenc}
\usepackage{xltxtra}
\usepackage{xecyr}

%\usepackage{polyglossia}  %% подключает пакет многоязыкой вёрстки
\setmainfont{Liberation Serif}  %% задаёт основной шрифт документа
\setsansfont{Liberation Sans}  %% задаёт шрифт без засечек
\setmonofont{Droid Sans Mono}  %% задаёт моноширинный шрифт
\defaultfontfeatures{Scale=MatchLowercase, Mapping=tex-text}  %% устанавливает поведение шрифтов по умолчанию
%\setdefaultlanguage[spelling=modern]{russian}  %% устанавливает язык по умолчанию
%\setotherlanguage{english}

\usepackage{cmap} % чтобы был поиск по pdf
\usepackage{ucs}
\usepackage{hyperref}

% меняем размеры страницы
\usepackage{geometry}
\geometry{left=2.0cm}
\geometry{right=2.0cm}
\geometry{top=2.0cm}
\geometry{bottom=2.5cm}

% переопределяем окружение verbatim
\makeatletter 
\renewcommand*\verbatim@font{% 
\normalfont\ttfamily\small 
\hyphenchar\font\m@ne 
\let\do\do@noligs \verbatim@nolig@list 
} 
\makeatother

\renewcommand{\thesection}{\arabic{section}.} 
\renewcommand{\thesubsection}{\arabic{section}.\arabic{subsection}.} 
\renewcommand{\thesubsubsection}{\arabic{section}.\arabic{subsection}.\arabic{subsubsection}.} 

\begin{document}

\begin{titlepage}
\begin{center}
\vspace*{10cm}
{ \rm \Huge \textbf
	{
		\fontspec{Ubuntu}{print(gentoo.calculate\_linux)}
	}
}
\vfill 
{ \rm \large  \fontspec{Ubuntu}{2011}  }
\end{center}
\end{titlepage}

\tableofcontents

\newpage

\section{Введение}

Данный документ распространяется под лицензией CC-BY\footnote{Creative Commons - \href{http://creativecommons.org/licenses/by/3.0/legalcode}{http://creativecommons.org/licenses/by/3.0/legalcode}}.

\subsection{Gentoo}

Gentoo Linux (произносится: дженту, сленг - генту, гента) - достаточно популярный дистрибутив Linux с мощной и гибкой технологией Portage, которая совмещает в себе возможности конфигурирования, настройки, а также автоматизированную систему управления пакетами. Последняя создавалась под влиянием системы управления пакетами в FreeBSD. Отличительной особенностью Gentoo является наличие оптимизации под конкретное аппаратное обеспечение.

Дистрибутив ведёт своё начало с разработки Дэниелом Роббинсом (англ.) дистрибутива Enoch Linux в 1999 году. Благодаря системе Portage, позволяющей сделать из Gentoo практически всё, что угодно (от сервера до рабочей станции), этот дистрибутив можно отнести к так называемым мета-дистрибутивам.

Gentoo (по-русски читается генту, название вида пингвинов укрепилось в русском языке задолго до появления дистрибутива) - это английское название вида пингвинов Pygoscelis papua. Согласно Книге рекордов Гиннеса 98, этому виду пингвинов принадлежит рекорд скорости плавания (36 км/ч).

Gentoo появился на основе разработок собственного дистрибутива Даниэла Роббинса под названием Enoch Linux. Основной целью дистрибутива являлась собственная сборка (подобная Linux From Scratch), оптимизированная под конкретное аппаратное обеспечение и содержащее минимальный, необходимый пользователю, набор программ.



Эксперименты над сборками GCC показали, что производительность получаемого пакета увеличивалась (в зависимости от аппаратного обеспечения) от 10 до 200 \% по сравнению с бинарными сборками gcc, поставляемыми другими дистрибутивами. Наработки, увеличивающие производительность, были включены в официальный выпуск GCC 2.95, благодаря чему другие дистрибутивы также получили дополнительный прирост в производительности. Начиная с этого момента, Enoch начал приобретать репутацию быстрого дистрибутива, после чего было принято решение сменить название на Gentoo Linux.

31 марта 2002 года вышла первая версия дистрибутива. В 2004 году была основана некоммерческая организация Gentoo Foundation, в ведение которой Роббинс передал все права на исходный код и торговые марки, тем самым отстранившись от разработки дистрибутива. На данный момент разработка управляется группой Board of Trustees из пяти человек (ответственных за управление фондом) и советом Gentoo Council (технические вопросы), состоящим из семи членов, выбираемых на срок в один год. В начале 2009 года обновился дизайн сайта. На главной странице отображаются новости, обсуждения и статьи разработчиков, а также извещения о критических уязвимостях (Gentoo Linux Security Advisory). В октябре 2009 года Gentoo исполнилось десять лет, в связи с чем подготовлена большая коллекция скриншотов и хронология дистрибутива.

Философия Gentoo проистекает от дней основания дистрибутива и остаётся в неизменном виде и по сей день. Вот как вкратце излагает концепцию проекта его основатель Дэниэл Роббинс:

\begin{center}
\begin{tabular}{p{14cm}}
Каждому пользователю приходится выполнять определённую работу. Цель Gentoo — разработка инструментов и систем, позволяющих пользователю заниматься своим делом как можно эффективнее и в своё удовольствие, так, как он сочтет нужным. Наши инструменты должны приносить радость и помогать пользователю оценить по достоинству все богатство Linux и сообщества свободного программного обеспечения, а также гибкость свободных программ. Такое возможно только тогда, когда инструменты создаются, чтобы отражать и проводить волю пользователя, оставляя для него открытыми все возможности с самого начала (с исходного кода). Когда инструмент заставляет пользователя действовать определённым образом, инструмент работает против него, а не на него. Все мы сталкивались с ситуациями, когда инструменты стремились навязать нам свою собственную волю. Такой подход — полная противоположность, несовместимая с философией Gentoo.
Иными словами, философия Gentoo — создание лучших инструментов. Когда инструмент в совершенстве выполняет свою задачу, вы можете даже не замечать его присутствия, потому что он не перечит вам, не проявляет себя, и не заставляет вас возиться с ним, когда вам совершенно не до этого. Инструмент служит пользователю, а не пользователь — инструменту.
Будущая задача Gentoo — продолжать борьбу за создание инструментов, близких к идеалу. Инструментов, удовлетворяющих нужды множества различных пользователей (каждого — со своими разнообразными целями) с простотой, идущей рука об руку с непревзойдённой мощью. Разве вы не любите пользоваться инструментами, которые отлично подходят для ваших нужд? Разве это не великолепное ощущение? Наша цель — передать это чувство как можно большему числу людей.
\newline
\begin{flushright}
— Дэниел Роббинс
\end{flushright}

\end{tabular}
\end{center}

\subsection{Calculate Linux}

\newpage

\section{Установка. Обновление. Профили}

\subsection{Системные требования}
\subsection{Gentoo}

\subsubsection{Gentoo. Установка}

Для установки необходимо скачать установочный диск с минимальным набором программ: 
install-amd64-minimal-[...].iso
(\href{http://mirror.yandex.ru/gentoo-distfiles/releases/amd64/current-iso/install-amd64-minimal-20110811.iso}{install-amd64-minimal-20110811.iso})

После того, как установили настройки в биосе, о том что необходимо произвести загрузку с cdrom, грузимся. После загрузки нам предлагают выбрать какое ядро грузить, пишем gentoo. Далее идет процесс загрузки ядра и в результате получаем консоль с правами доступа root.


{\bf Подготавиваем HDD}


Для установки потребуется жесткий диск размером не менее 6,5 гб. Разбить диск необходимо следующим образом:

\begin{center}
\begin{tabular}{|l|l|l|l|}
\hline точка монтирования  & размер & файловая система & раздел диска \\
\hline swap  & 2 * RAM & swap & /dev/sda1 \\
\hline /boot & 200 Mb & ext2  & /dev/sda2 \\
\hline / & минимум 6 Гб & ext4 & /dev/sda3 \\
\hline
\end{tabular}
\end{center}

Что бы произвести разметку диска, необходимо воспользоваться программой fdisk:

\begin{verbatim}
# fdisk /dev/sda
\end{verbatim}

% TODO: написать о том как пользоваться программой fdisk

После того как разделы были созданы, необходимо их отформатировать, с указанием файловой системы:

\begin{verbatim}
# mkfs.ext2 /dev/sda2
# mkfs.ext4 /dev/sda4
\end{verbatim}

Создадим swap и запустим его:

\begin{verbatim}
# mkswap /dev/sda1
# swapon /dev/sda1
\end{verbatim}

{\bf Устанавливаем разделы}

\begin{verbatim}
# mount /dev/sda3 /mnt/gentoo
# mkdir /mnt/gentoo/boot
# mount /dev/sda2 /mnt/gentoo/boot
\end{verbatim}

Далее необходимо скачать последние пакеты (stage3-i686-xxx.tar.bz2, portage-latest.tar.bz2)



\subsubsection{Gentoo. Обновление}

\subsection{Calculate Linux}

\subsubsection{Calculate Linux. Установка}
\subsubsection{Calculate Linux. Обновление}

\subsection{Управление профилями пользователей}

\newpage 

\section{Работа с командной строкой}

\subsection{Поиск файлов}

\newpage

\section{Менеджер пакетов}

\subsection{Установка и удаление программ}

\subsection{Разрешение зависимостей}

\subsection{Написание собственных ebuild файлов}

\newpage
\section{Настройка}

\subsection{Настройка сети}

\subsection{Настройка авто запуска}

%\subsection{}
%\subsection{}
%\subsection{}
%\subsection{}
%\subsection{}
%\subsection{}
%\subsection{}

\subsection{Персональные настройки}

\subsubsection{Настройка bash}

\subsubsection{Настройка vim}

\newpage
\section{Мониторинг и производительность}

\subsection{Мониторинг}

\subsection{Оптимизация eix}

\subsection{SSD}

\subsection{Распределенная компиляция}

\newpage

\section{Приложение}

%\subsection{}

\newpage
\begin{thebibliography}{99}
\addcontentsline{toc}{section}{Список литературы}

	\bibitem{wiki_gentoo} http://ru.wikipedia.org/wiki/Gentoo
	
	

\end{thebibliography}

\end{document}

