\section{Обновление системы}
\subsection{Обновление из ISO-файла}
Вы можете загрузить последнюю версию iso-файла с образом системы и выполнить обновление. Для этого вам понадобится свободный раздел ~10Гб и 5-10 минут свободного времени. При этом вы можете продолжать работать в системе. Перед дальнейшими действиями, если вы обновляете сервер, обязательно прочтите руководство по резервному копированию данных.
Откройте консоль с правами пользователя root и выполните следующие действия:
\begin{enumerate}
\item Если у вас установлен Calculate Directory Server, удостоверьтесь, что директория /var/calculate монтируется с отдельного раздела жесткого диска. Если это не так, перенесите свои данные и добавьте точку монтирования в /etc/fstab.
\item Обновите calculate overlay, выполнив:
\begin{verbatim}
layman -S
\end{verbatim}

\item Обновите утилиту установки. Для установки Calculate Linux 10.9 и выше, выполните:
\begin{verbatim}
emerge calculate-install
\end{verbatim}
\item Загрузите ISO образ с последней версией обновления системы здесь, после чего выполните:
\begin{verbatim}
cd /usr/calculate/share/linux
wget ftp://ftp.gtlib.gatech.edu/pub/calculate/CLD/10.2/i686/cld-10.2-i686.iso
\end{verbatim}
Подставьте правильный путь к файлу с образом для вашего дистрибутива нужной архитектуры. Узнать свою архитектуру можно выполнив команду:
arch
\item Установите новую версию системы. Для установки Calculate Linux 10.9 и выше, выполните:
\begin{verbatim}
cl-install
\end{verbatim}
Для установки Calculate Linux 10.4.x, выполните:
\begin{verbatim}
calculate
\end{verbatim}
В случае разбиения диска по предложенной схеме, дополнительных параметров для выполнения команды не потребуется. При переустановке той же версии системы, потребуется добавить параметр ''-f''.
\item Если вы обновляете Calculate Directory Server, сохраните копию настроек сервисов и базы LDAP, выполнив:
\begin{verbatim}
cl-backup
\end{verbatim}
\item Перезагрузите компьютер.
\item Для восстановления LDAP базы и настроек сервера, выполните:
\begin{verbatim}
cl-rebuild
\end{verbatim}
\end{enumerate}

\subsection{Утилиты обновления системы}
Операционные системы Linux отличаются своей надежностью и защищенностью. Тем не менее пренебрегать обновлением не стоит. Данное руководство описывает шаги, которые выполняются при обновлении пакетов системы.
\subsubsection{Обновление портежей и оверлея}
Для обновления пакетов используйте утилиту emerge. Перед обновлением, выполните обновление портежей и оверлея Calculate. Для этого выполните в консоли с правами root:
eix-sync
После этого вы можете приступить к обновлению пакетов.
\subsubsection{Обновление основных пакетов}
Под основными пакетами подразумеваются пакеты, описанные в файле world. Файл имеет текстовый формат, находится в директории /var/lib/portage/, и заполняется по мере установки новых пакетов. К примеру если вы устанавливаете пакет games-strategy/freeciv и он по зависимостям устанавливает еще несколько библиотек, в world попадет только этот пакет. То есть тот пакет, который вы непосредственно укажете утилите emerge.
Для отображения списка пакетов подлежащих обновлению, выполните:
\begin{verbatim}
emerge -up world
\end{verbatim}
Установить программы можно убрав параметр ''p'', командой:
\begin{verbatim}
emerge -u world
\end{verbatim}
\subsubsection{Обновление всех пакетов}
Вы можете выполнить обновление не только основных пакетов, но и всех пакетов в системе. Для этого выполните:
\begin{verbatim}
emerge -uD system
emerge -uD world
\end{verbatim}
Перед выполнением неплохо взглянуть на список обновляемых пакетов. Для этого как и в предыдущем примере, добавьте опцию ''p'' к команде.
В приведенном примере обновляются пакеты system и world. В system входят базовые пакеты системы, такие как baselayout, gcc, glibc и прочие.
\subsubsection{Размаскировка пакетов}
Может так случиться, что обновляемый пакет потребует в зависимостях замаскированный для установки пакет. Разрешить зависимости поможет утилита cl-unmask. Для этого просто выполните
\begin{verbatim}
cl-unmask world
\end{verbatim}
\subsubsection{Проверка зависимостей}
Обновляя пакеты системы, особенно применяя опцию ''D'' для полного обновления, могут обновиться библиотеки, используемые другими программами. Что может привести к неработоспособности последних. К счастью, существует замечательная утилита revdep-rebuild, проверяющая все зависимости, и, в случае обнаружения проблем, переустанавливающая необходимые пакеты. Достаточно выполнить команду:
\begin{verbatim}
revdep-rebuild
\end{verbatim}

\subsubsection{Обновление конфигурационных файлов}
Новые версии программ могут потребовать новых конфигурационных файлов. Особенно это актуально для файлов, входящих в директорию /etc/init.d/. Здесь находятся скрипты запуска сервисов. Если их не обновлять при установке новых версий программ, сервисы могут оказаться в нерабочем состоянии
Заменять конфигурационные файлы во время установки обновлений также опасно. Так как в процессе настройки вы могли их изменить. 
Поэтому файлы настроек сохраняются под новыми именами, добавляя к началу имени строку ''.\_cfg0000\_''.
Вы можете найти и заменить существующие, либо удалить новые конфигурационные файлы. Для этого выполните команду:
\begin{verbatim}
dispatch-conf
\end{verbatim}
Относитесь к ней крайне осторожно! Обратите внимание на изменений файлов, отмеченных в начале как \# Changed by Calculate. Такие файлы были настроены утилитой calculate и замена их может привести к отключению некоторых настроек.
Основные команды: ''PageUp''/''PageDown'' - перемещаться по файлу, ''u'' - заменить существующий файл новым, ''z'' - удалить новый конфигурационный файл, ''q'' - прервать работу.